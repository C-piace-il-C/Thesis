% Chapter Template

\chapter{L'algoritmo H265} % Main chapter title

\label{Chapter4} % Change X to a consecutive number; for referencing this chapter elsewhere, use \ref{ChapterX}

\lhead{Capitolo 4. \emph{L'algoritmo H265}} % Change X to a consecutive number; this is for the header on each page - perhaps a shortened title

%----------------------------------------------------------------------------------------
%	SECTION 1
%----------------------------------------------------------------------------------------

\section{Suddivisione in blocchi dell'immagine} 
Un'immagine viene inizialmente suddivisa in \emph{coding tree unit} (CTU), di 
forma quadrata e dimensione costante per tutta la sequenza video: 64x64, 32x32 
o 16x16 pixel. Questa ``flessibilità'' nella dimensione del blocco fondamentale 
della suddivisione è stata introdotta da HEVC, in quanto tutti i suoi 
predecessori utilizzano un \emph{macroblocco} di 16x16 pixel; ciò permette a 
HEVC di sapersi adattare meglio a -e comprimere maggiormente- video di diverse 
dimensioni.
Il CTU è un'unità logica che consiste di tre ulteriori blocchi: Luma, 
ChromaB e ChromaR (Y, Cb e Cr). Ognuno di questi blocchi è un 
\emph{coding tree block} (CTB), ed è della stessa dimensione relativa 
del CTU, sebbene la dimensione effettiva di ogni blocco sia regolata dal 
\emph{chroma sampling format}: se il formato fosse 4:2:0, tipico di questi 
encoding, il blocco Y sarebbe 64x64 pixel, mentre i due blocchi di crominanza 
risulterebbero di 32x32 pixel. Quando un blocco risulta più piccolo del 
CTU di partenza (in questo caso, e quasi sempre, i due di crominanza) 
subisce un \emph{upscaling}: questo comporta una minore definizione e una 
maggiore compressione dell'immagine finale, resa ammissibile dalla maggiore 
sensibilità dell'apparato visivo umano alla luminanza rispetto al colore.
// Qui servirebbe un'immagine
Il CTB può essere ulteriormente suddiviso in \emph{coding blocks} (CB), che 
sono il punto in cui viene decisa quale tipo di \emph{prediction} utilizzare.
Supponendo di avere un CTB di dimensione 64x64 la suddivisione può essere 
effettuata con CB grandi 64x64, 32x32, 16x16 o 8x8: nello stesso gruppo di 
CTB la dimensione dei CB deve essere omogenea, mentre può differire tra CTB di 
CTU diversi. Il CB, così come il CTB, consiste ancora nei tre blocchi Y, Cb e 
Cr, che definiscono un \emph{coding unit} (CU), ovverò l'unità in cui viene 
codificato il tipo di predizione. La scelta di quest'ultima è autonoma per ogni 
CU.
// E qui ne servirebbe un'altra



%----------------------------------------------------------------------------------------
%	SECTION 2
%----------------------------------------------------------------------------------------

\section{Encoder}

%-----------------------------------
%       SUBSECTION 1
%-----------------------------------

\subsection{Motion Estimation}

%-----------------------------------
%       SUBSECTION 2
%-----------------------------------

\subsection{Transform, Scaling \& Quantization}

%-----------------------------------
%       SUBSECTION 3
%-----------------------------------

\subsection{CABAC Entropy Coding}

%----------------------------------------------------------------------------------------
%       SECTION 3
%----------------------------------------------------------------------------------------

\section{Decoder}

%-----------------------------------
%       SUBSECTION 1
%-----------------------------------

\subsection{Intra Prediction}

%-----------------------------------
%       SUBSECTION 2
%-----------------------------------

\subsection{Inter Prediction}

%-----------------------------------
%       SUBSECTION 3
%-----------------------------------

\subsection{Scaling \& Inverse Transform}

%-----------------------------------
%       SUBSECTION 4
%-----------------------------------

\subsection{In-Loop Filter}

%-----------------------------------
%       SUBSECTION 5
%-----------------------------------

\subsection{DPB \& Output}

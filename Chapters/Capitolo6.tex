% Chapter 

\chapter{Il progetto} % Main chapter title

\label{Chapter6} % Change X to a consecutive number; for referencing this chapter elsewhere, use \ref{ChapterX}

\lhead{Capitolo 6. \emph{Il progetto}} % Change X to a consecutive number; this is for the header on each page - perhaps a shortened title

%----------------------------------------------------------------------------------------
%	SECTION 1
%----------------------------------------------------------------------------------------

\section{Il setup della piattaforma}
La prima decisione che ha riguardato la piattaforma è stato il sistema 
operativo da installare. Vi sono a disposizione diversi sistemi 
Unix-like\footnote{http://www.lemaker.org/portal.php?mod=list\&catid=4}, 
dai più personalizzabili (come Gentoo o Arch Linux) a quelli più
 \emph{user-friendly} (come Lubuntu); la nostra scelta è ricaduta su Bananian, 
una distribuzione che deriva da Debian 7, appositamente ottimizzata per il 
Banana Pi, che abbiamo ritenuto essere il giusto compromesso tra usabilità
e assenza di software preinstallato a noi non necessario. \\
Nonostante la possibilità di lavorare senza ambiente grafico, da terminale, 
risparmiando qualche decina di \emph{megabyte} di RAM, inizialmente abbiamo 
deciso di avvalerci di LXDE, un ambiente desktop estremamente leggero, in modo da 
velocizzare la maggior parte delle nostre interazioni con il sistema. \\
\\ \\
Essendo tre persone a lavorare su una sola piattaforma, abbiamo optato 
per collocare la \emph{board} in un laboratorio dell'università, dotandola
di un sistema di controllo remoto basato su \emph{virtual network computing} 
(VNC), un sistema che utilizza un protocollo \emph{remote framebuffer} (RBF) 
per l'accesso remoto alle interfacce grafiche utente (meglio conosciute come 
\textbf{GUI}s, \textbf{G}raphical \textbf{U}ser \textbf{I}nterfaces). Inoltre, 
per facilitare l'accesso ai dati presenti sulle periferiche di memorizzazione, 
abbiamo dotato il Banana Pi di un server \emph{file transfer protocol} (FTP).
\\ \\ 
Tuttavia, nella fase finale del progetto, l'intefaccia grafica è stata 
abbandonata in modo da permettere una risposta del sistema più immediata e una 
minore allocazione di risorse; è stato dunque possibile disattivare 
l'accelerazione hardware, necessaria in un ambiente grafico, permettendo il 
risparmio di circa 30MB di RAM. Insieme alla GUI è stata accantonato il 
controllo remoto tramite VNC (fortemente dispendioso a causa del suo approccio 
\emph{pixel-based}) a favore di un accesso tramite SSH (\emph{secure shell}), 
più essenziale e meno dispersivo.

%----------------------------------------------------------------------------------------
%	SECTION 2
%----------------------------------------------------------------------------------------

\section{I tool per valutare le prestazioni}
Per la valutazione delle prestazioni degli encoder da ottimizzare, è stato 
fatto uso di due strumenti software:
\begin{itemize}
\item Valgrind
\item gperftools
\end{itemize}
Entrambi \emph{opensource}, offrono diversi strumenti per l'analisi dinamica 
di un software; quelli utilizzati in questo progetto sono stati
\emph{Cachegrind} e \emph{Callgrind} per quanto riguarda il primo e il 
\emph{CPU profiler} del secondo.
%----------------------------------------------------------------------------------------
%	SECTION 3
%----------------------------------------------------------------------------------------

\section{I software di encoding presi in considerazione}
Inizialmente, tre software di encoding sono stati individuati come possibili
candidati per il progetto:
\begin{itemize}
\item f265
\item HM (\textbf{H}EVC Test \textbf{M}odel)
\item x265
\end{itemize}
Il primo è stato subito scartato non appena è stato chiaro che il sorgente non 
è compatibile con la piattaforma in nostro possesso (il codice è stato scritto 
unicamente per architettura x86).
I rimanenti due sono stati oggetto di 
%----------------------------------------------------------------------------------------
%	SECTION 4
%----------------------------------------------------------------------------------------

\section{Test preliminari}

%----------------------------------------------------------------------------------------
%	SECTION 5
%----------------------------------------------------------------------------------------

\section{Individuazione dei moduli H.265}

%-----------------------------------
%       SUBSECTION 1
%-----------------------------------

\subsection{Il tool KCacheGrind / QCacheGrind}

%----------------------------------------------------------------------------------------
%	SECTION 6
%----------------------------------------------------------------------------------------

\section{Strategie di ottimizzazione}

%-----------------------------------
%       SUBSECTION 1
%-----------------------------------

\subsection{Compilation flags}

%-----------------------------------
%       SUBSECTION 2
%-----------------------------------

\subsection{Assembly}

%-----------------------------------
%       SUBSECTION 3
%-----------------------------------

\subsection{NEON intrinsics}

%-----------------------------------
%       SUBSECTION 4
%-----------------------------------

\subsection{Switch}

%-----------------------------------
%       SUBSECTION 5
%-----------------------------------

\subsection{Attributes}

%-----------------------------------
%       SUBSECTION 6
%-----------------------------------

\subsection{Multithread}


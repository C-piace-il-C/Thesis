% Chapter 

\chapter{Il progetto} % Main chapter title

\label{Chapter6} % Change X to a consecutive number; for referencing this chapter elsewhere, use \ref{ChapterX}

\lhead{Capitolo 6. \emph{Il progetto}} % Change X to a consecutive number; this is for the header on each page - perhaps a shortened title

%----------------------------------------------------------------------------------------
%	SECTION 1
%----------------------------------------------------------------------------------------

\section{Il setup della piattaforma}
La prima decisione che ha riguardato la piattaforma è stato il sistema 
operativo da installare. Vi sono a disposizione diversi sistemi 
Unix-like\footnote{http://www.lemaker.org/portal.php?mod=list\&catid=4}, 
dai più personalizzabili (come Gentoo o Arch Linux) a quelli più
 \emph{user-friendly} (come Lubuntu); la nostra scelta è ricaduta su Bananian, 
una distribuzione che deriva da Debian 7, appositamente ottimizzata per il 
Banana Pi, che abbiamo ritenuto essere il giusto compromesso tra usabilità
e assenza di software preinstallato a noi non necessario. \\
Nonostante la possibilità di lavorare senza ambiente grafico, da terminale, 
risparmiando qualche decina di \emph{megabyte} di RAM, inizialmente abbiamo 
deciso di avvalerci di LXDE, un ambiente desktop estremamente leggero, in modo da 
velocizzare la maggior parte delle nostre interazioni con il sistema. \\
\\ \\
Essendo tre persone a lavorare su una sola piattaforma, abbiamo optato 
per collocare la \emph{board} in un laboratorio dell'università, dotandola
di un sistema di controllo remoto basato su \emph{virtual network computing} 
(VNC), un sistema che utilizza un protocollo \emph{remote framebuffer} (RBF) 
per l'accesso remoto alle interfacce grafiche utente (meglio conosciute come 
\textbf{GUI}s, \textbf{G}raphical \textbf{U}ser \textbf{I}nterfaces). Inoltre, 
per facilitare l'accesso ai dati presenti sulle periferiche di memorizzazione, 
abbiamo dotato il Banana Pi di un server \emph{file transfer protocol} (FTP).
\\ \\ 
Tuttavia, nella fase finale del progetto, l'intefaccia grafica è stata 
abbandonata in modo da permettere una risposta del sistema più immediata e una 
minore allocazione di risorse; è stato dunque possibile disattivare 
l'accelerazione hardware, necessaria in un ambiente grafico, permettendo il 
risparmio di circa 30MB di RAM. Insieme alla GUI è stata accantonato il 
controllo remoto tramite VNC (fortemente dispendioso a causa del suo approccio 
\emph{pixel-based}) a favore di un accesso tramite SSH (\emph{secure shell}), 
più essenziale e meno dispersivo.

%----------------------------------------------------------------------------------------
%	SECTION 2
%----------------------------------------------------------------------------------------

\section{I tool per la compilazione}

%-----------------------------------
%       SUBSECTION 1
%-----------------------------------

\subsection{Il compilatore GCC}

%-----------------------------------
%       SUBSECTION 2
%-----------------------------------

\subsection{Il tool make}

%----------------------------------------------------------------------------------------
%	SECTION 3
%----------------------------------------------------------------------------------------

\section{I tool per valutare le prestazioni}
Per la valutazione delle prestazioni degli encoder da ottimizzare, è stato 
fatto uso di due strumenti software:
\begin{itemize}
\item Valgrind
\item gperftools
\end{itemize}
Entrambi \emph{opensource}, offrono diversi strumenti per l'analisi dinamica 
di un software; quelli utilizzati in questo progetto sono stati
\emph{Cachegrind} e \emph{Callgrind} per quanto riguarda il primo e il 
\emph{CPU profiler} del secondo.
%----------------------------------------------------------------------------------------
%	SECTION 3
%----------------------------------------------------------------------------------------

\section{I software di encoding presi in considerazione}
Inizialmente, tre software di encoding sono stati individuati come possibili
candidati per il progetto:
\begin{itemize}
\item f265
\item HM (\textbf{H}EVC Test \textbf{M}odel)
\item x265
\end{itemize}
Il primo è stato subito scartato non appena è stato chiaro che il sorgente non 
è compatibile con la piattaforma in nostro possesso (il codice è stato scritto 
unicamente per architettura x86).
I rimanenti due sono stati oggetto di 
%----------------------------------------------------------------------------------------
%	SECTION 4
%----------------------------------------------------------------------------------------

\section{Test preliminari}

%----------------------------------------------------------------------------------------
%	SECTION 5
%----------------------------------------------------------------------------------------

\section{Individuazione dei moduli H.265}

%-----------------------------------
%       SUBSECTION 1
%-----------------------------------

\subsection{Il tool KCacheGrind / QCacheGrind}

%----------------------------------------------------------------------------------------
%	SECTION 6
%----------------------------------------------------------------------------------------

\section{Strategie di ottimizzazione}

%-----------------------------------
%       SUBSECTION 1
%-----------------------------------

\subsection{Compilation flags}
Per prima cosa abbiamo dedicato del tempo allo studio delle opzioni di 
compilazione fornite da GCC (\emph{GNU Compiler Collection}) al fine di partire 
da una ``base stabile" da ottimizzare.\\
Particolare attenzione è stata dedicata ai \emph{flag} dedicati al 
miglioramento delle performance ed a quelli specifici per ARM.\\
\\
Senza alcuna opzione di compilazione espressa l'obiettivo del compilatore è 
quello di ridurre il più possibile il costo della compilazione e di rendere al 
contempo praticabile il \emph{debug} del programma. Per rendere possibile il 
//debug ciò è necessario in primo luogo che ogni \emph{statement} sia 
indipendente: è possibile fermare l'esecuzione in qualsiasi punto utilizzando 
un \emph{breakpoint} al fine di assegnare a piacere valori alle variabili e/o 
di modificare il \emph{program counter}, ottenendo i risultati attesi dal 
codice.\\
\\
Il compilatore ottimizza il codice basandosi sulla conoscenza che ha del 
programma. Non tutte le ottimizzazioni sono controllabili direttamente via 
\emph{flag}.\\
\\
Passiamo ora ad una breve descrizione delle opzioni utilizzate e dei loro 
effetti sull'eseguibile generato.

Famiglia `-O': opzioni dedicate all'ottimizzazione delle performance e/o della 
dimensione del compilato. La `O' è un diminutivo di ``Optimize".\\
Si distinguono vari livelli di ottimizzazione messi a disposizione dal 
compilatore GCC:
\begin{itemize}
\item \verb|-O0|\\
Opzione predefinita: riduce il tempo di compilazione cercando di dare la 
migliore esperienza di \emph{debug} possibile.
\item \verb|-O / -O1|\\
Abilita tutte le opzioni specificate da \verb|-O0|.\\
Il compilatore cerca di ridurre la dimensione del compilato ed il tempo di 
esecuzione utilizzando \emph{flag} che non peggiorano drasticamente il tempo 
di compilazione.
\item \verb|-O2|\\
Abilita tutte le opzioni specificate da \verb|-O / -O1|.\\
Vengono eseguite tutte le ottimizzazioni che non coinvolgono un 
\emph{trade-off} spazio-velocità. Rispetto al precedente migliora le 
prestazioni allungando il tempo di compilazione.
\item \verb|-Os|\\
Abilita tutte le opzioni specificate da \verb|-O2| che tipicamente non 
aumentano la dimensione dell'eseguibile. Vengono eseguite tutte le 
ottimizzazioni a favore dello spazio occupato dal compilato. La `s' è un 
diminutivo di ``size".
\item \verb|-O3|\\
Abilita tutte le opzioni specificate da \verb|-O2|.\\
Vengono eseguite tutte le ottimizzazioni a favore della velocità di esecuzione. 
Il tempo di compilazione aumenta così come lo spazio occupato dall'eseguibile 
generato. E' il massimo livello di ottimizzazione possibile insieme ad 
\verb|-Ofast|.
\item \verb|-Ofast|\\
Abilita tutte le opzioni specificate da \verb|-O3|.\\
Ignora l'adesione rigorosa allo standard abilitando \verb|-ffast-math|.
\item \verb|-Og|\\
Ottimizza l'esperienza di \emph{debug} abilitando tutte le opzioni che non 
interferiscono con quest'ultimo.
\end{itemize}

Opzioni di ottimizzazione indipendenti dall'hardware:

\begin{itemize}
\item \verb|-ftree-vectorize|\\
Abilita la vettorizzazione dei \emph{tree}.
// Che cos'è un tree?\\
Il compilatore cerca di riorganizzare dati in vettori permettendo così 
l'utilizzo di istruzioni SIMD (\emph{Single Instruction Multiple Data}) al fine 
di migliorare il tempo di esecuzione del codice.\\
Essa abilita inoltre \verb|-ftree-loop-vectorize| e 
\verb|-ftree-slp-vectorize|, che abilitano rispettivamente la vettorizzazione 
dei \emph{loop} e dei \emph{basic block}.
\item \verb|-finline-functions|\\
Considera tutte le funzioni come candidate per un possibile \emph{inlining}, 
anche quelle che non sono dichiarate esplicitamente come \verb|inline|.\\
Il compilatore decide attraverso un calcolo euristico di effettuare o no 
l'\emph{inlining} della funzione. 
\item \verb|-funswitch-loops|\\
Abilitata automaticamente con l'opzione \verb|-O3|.\\
Sposta i \emph{branch} con condizioni che non dipendono dal \emph{loop} al di 
fuori di quest'ultimo, duplicandolo su entrambi i \emph{branch} e modificandolo 
tenendo conto del risultato della condizione.
\item \verb|-funroll-loops|\\
Abilita l'\emph{unrolling} dei \emph{loop} per i quali è possibile determinare 
il numero di iterazioni a \emph{compile time}.
// Aggiungere -fwhole-program?
\end{itemize}

Opzioni di ottimizzazione specifiche per l'hardware ARM:

\begin{itemize}
\item \verb|-march=|\emph{name}\\
Specifica il nome dell'architettura ARM di destinazione.\\
Questo parametro viene utilizzato da GCC per determinare che tipo di istruzioni 
possono essere emesse quando viene generato il codice assembly relativo al 
programma.
\item \verb|-mtune=|\emph{name}\\
Specifica il nome del processore ARM per il quale GCC deve ottimizzare il 
codice. Su alcune implementazioni possono essere ricavate prestazioni migliori 
specificando questa opzione.
\item \verb|-mfpu=|\emph{name}\\
Specifica quale hardware (o emulazione hardware) \emph{floating-point} è 
disponibile sul dispositivo.\\
Quest'opzione è \underline{necessaria} per poter utilizzare le varie SIMD (in 
questo caso NEON) messe a disposizione dall'architettura ARM.
\item \verb|-mfloat-abi=|\emph{name}\\
Specifica quale ABI (o \emph{Application Binary Interface}) utilizzare per le 
operazioni \emph{floating-point}. // Che cos'è un ABI? \\
Quest'opzione è \underline{necessaria} per poter utilizzare le varie SIMD (in 
questo caso NEON) messe a disposizione dall'architettura ARM.
\end{itemize}

Opzioni del linguaggio e del \emph{debug}.

\begin{itemize}
\item \verb|-fopt-info-|\emph{options}\\
Mostra un \emph{log} contenente informazioni su ciò che è o non è stato 
ottimizzato.
\item \verb|-std=|\emph{name}\\
Determina quale standard utilizzare per il linguaggio C da compilare.
\item \verb|-pthread|\\
Abilita il supporto al \emph{multithreading} con la libreria \emph{phtreads}.
\end{itemize}

Queste opzioni sono state testate modificando il file \verb|makefile.base|, 
contenuto nella cartella \verb|/build/linux/common/|. Di seguito le linee 
dall'originale:\\

\begin{lstlisting}[language=make]
# default cpp flags for all configurations
CPPFLAGS          = -fPIC $(DEFS) -I$(CURDIR)/$(INC_DIR) $(USER_INC_DIRS) -Wall
                    -Wshadow -Wno-sign-compare -Werror
# debug cpp flags
DEBUG_CPPFLAGS    = -g -D_DEBUG
# release cpp
RELEASE_CPPFLAGS  = -O3 -Wuninitialized
\end{lstlisting}

Per prima cosa è stato cambiato il livello di ottimizzazione da \verb|-O3| a 
\verb|-Ofast| per quanto riguarda i \verb|RELEASE_CPPFLAGS|.\\
Questa modifica è stata effettuata al fine di migliorare le \emph{performance} 
sulle operazioni matematiche, è stato inoltre verificato che il comportamento 
del programma non fosse cambiato (\verb|-ffast-math| può portare a risultati 
sbagliati in certe configurazioni).\\
Assieme a suddetta modifica è stato cambiato il \emph{flag} \verb|-g| in 
\verb|-Og| per quanto riguarda i \verb|DEBUG_CPPFLAGS|.\\
Questo al fine di velocizzare il più possibile il \emph{debug} 
dell'applicazione, particolarmente lento utilizzando \verb|-g|.\\
Sono state poi inserite tutte le opzioni relative all'hardware ARM in 
\verb|CPPFLAGS|, i \emph{flag} condivisi da tutte le configurazioni.\\
Nello specifico sono state inserite le seguenti voci: \verb|-march=armv7-a|, 
\verb|-mtune=cortex-a7|, \verb|-mfpu=neon|, \verb|-mfloat-abi=softfp|.\\
Subito dopo questa modifica è stato ancora aggiunta ai \verb|RELEASE_CPPFLAGS| 
l'opzione \verb|-ftree-vectorize|.\\
\\
Prima di iniziare la modifica vera e propria del codice è stata effettuata 
un'analisi mediante l'utilizzo di \verb|-fopt-info-vec-optimized| al fine di 
sapere cosa fosse stato già vettorizzato automaticamente da GCC. Questo ci ha 
permesso di focalizzarci maggiormente sulle funzioni onerose non modificate.\\
\\
E' stato inoltre deciso di rendere \emph{multithread} il programma.\\
E' stato in primo luogo provato il \emph{multithreading} offerto dallo standard 
C++11, aggiungendo quindi \verb|-std=c++11| alle opzioni esistenti.\\
Avendo però notato un degrado delle performance generali in \emph{single 
thread}, sì è deciso di utilizzare la libreria \emph{pthreads} e quindi di 
sostituire il \emph{flag} \verb|-std=c++11| con \verb|-pthread|.

Il file \verb|makefile.base| finale è quindi il seguente:\\

\begin{lstlisting}[language=make]
# default cpp flags for all configurations
CPPFLAGS          = -fPIC $(DEFS) -I$(CURDIR)/$(INC_DIR) $(USER_INC_DIRS) -Wall 
                    -Wshadow -Wno-sign-compare -Werror -march=armv7-a 
                    -mtune=cortex-a7 -mfpu=neon -mfloat-abi=softfp -pthread
# debug cpp flags
DEBUG_CPPFLAGS    = -Og -D_DEBUG
# release cpp
RELEASE_CPPFLAGS  = -Ofast -Wuninitialized -ftree-vectorize
\end{lstlisting}


// Tentativi con -fwhole-program.\\
// gcc-4.7 non vettorizzava ARM, gcc-4.9 sì.\\
// Risultati\\

%-----------------------------------
%       SUBSECTION 2
%-----------------------------------

\subsection{Assembly}
// IMPOSTA FONT CONSOLAS PER I LISTATI \newline
// INGRANDISCI IL FONT DI TESTO E LISTATI \newline
Una delle prime ottimizzazioni tentate durante sviluppo del lavoro è stata
 la stesura di funzioni in Assembly nel tentativo di ottimizzare alcune parti 
 di codice delle funzioni più onerose di HM.
Le principali caratteristiche del linguaggio Assembly ARM che si possono sfruttare per ottimizzare sono le istruzioni condizionali e il \emph{barrel shifter}. \newline
Le istruzioni condizionali sono normali istruzioni la cui esecuzione dipende dal valore di determinati \emph{flag}. In questo modo è possibile evitare parecchi branch condizionali che rallentano il codice perché rompono la \emph{pipeline}.\newline
Il \emph{barrel shifter} è un'unità che permette la pre-elaborazione di uno dei 
due operandi di un'istruzione attraverso una normale operazione di shift. 
//AGGIUNGI IMMAGINE BARREL SHIFTER \newline
\paragraph*{Clamp} La funzione matematica $\text{clamp}(x) = 
\max(a,\min(x,b))$, assente dalla libreria standard, viene implementata da 
\verb+filter+ in questo modo:
\lstinputlisting[language=C,caption=]{Codes/cclamp.c}
Questo codice viene tradotto in \verb|-Ofast| in una dozzina di istruzioni con 
due \emph{compare}. Notando che nel codice di HM \verb|minVal| vale sempre 0, è 
possibile introdurre un'ottimizzazione che sfrutta il barrel shifter. \newline
Segue il confronto tra l'estratto del codice Assembly generato da GCC (a 
sinistra) e quello da noi scritto per eseguire clamp.
// ALLINEA VERTICALMENTE I DUE LISTATI
\lstset{style=cstyle}
\begin{center}
  \begin{tabular}{ l l }
    \lstinputlisting[language=C,caption=]{Codes/armgccclamp.s} &
    \lstinputlisting[language=C,stepnumber=0,caption=]{Codes/armclamp.s} \\
  \end{tabular}
\end{center}
Per maggiore chiarezza il codice generato da GCC è stato commentato riga per 
riga e viene discusso a seguire. \newline
Le prime due linee caricano 0 e \verb|val|, che si trova nella posizione 
puntata dallo \emph{stack pointer} + 4 rispettivamente nei registri \verb|r2| e 
\verb|r3|.\newline La terza linea esegue l'operazione \verb|r3 - r2| 
aggiornando i flag contenuti in \verb|CPSR| (Current Program Status 
Register).\newline L'istruzione \verb|itt lt| abilita la condizione \verb|lt| 
(lower than) per le successive due istruzioni, che verranno quindi eseguite 
solo se \verb|r3| $<$ \verb|r2|. Ciò si riflette sull'\emph{opcode} delle due 
istruzioni successive, che eredita il suffisso \verb|lt|. Queste istruzioni 
mettono 0 in \verb|r3| e lo salvano in \verb|val|. \newline Le restanti 
istruzioni eseguono il secondo \verb|if| in maniera analoga. \newline
Nel nostro codice la terza linea compendia quello che in C sarebbe 
\verb|if(r3 < 0) r3 = 0|. Il suffisso \verb|s| specifica che l'istruzione 
aggiorna anche il 
registro \verb|CPSR|, così, se il risultato è positivo (i.e. \verb|r3| $>$ 0), 
si procede con le istruzioni 5 e 6 che sostituiscono \verb|maxVal| a \verb|r3| 
nel caso in cui \verb|r3| $>$ \verb|maxVal|. 
%-----------------------------------
%       SUBSECTION 3
%-----------------------------------

\subsection{NEON intrinsics}

%-----------------------------------
%       SUBSECTION 4
%-----------------------------------

\subsection{Switch}

%-----------------------------------
%       SUBSECTION 5
%-----------------------------------

\subsection{Attributes}

%-----------------------------------
%       SUBSECTION 6
%-----------------------------------

\subsection{Multithread}


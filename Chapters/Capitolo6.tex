% Chapter 

\chapter{Il progetto} % Main chapter title

\label{Chapter6} % Change X to a consecutive number; for referencing this chapter elsewhere, use \ref{ChapterX}

\lhead{Capitolo 6. \emph{Il progetto}} % Change X to a consecutive number; this is for the header on each page - perhaps a shortened title

%----------------------------------------------------------------------------------------
%	SECTION 1
%----------------------------------------------------------------------------------------

\section{Il setup della piattaforma}
La prima decisione che ha riguardato la piattaforma è stato il sistema 
operativo da installare. Vi sono a disposizione diversi sistemi 
Unix-like\footnote{http://www.lemaker.org/portal.php?mod=list\&catid=4}, 
dai più personalizzabili (come Gentoo o Arch Linux) a quelli più
 \emph{user-friendly} (come Lubuntu); la nostra scelta è ricaduta su Bananian, 
una distribuzione che deriva da Debian 7, appositamente ottimizzata per il 
Banana Pi, che abbiamo ritenuto essere il giusto compromesso tra usabilità
e assenza di software preinstallato a noi non necessario. \\
Nonostante la possibilità di lavorare senza ambiente grafico, da terminale, 
risparmiando qualche decina di \emph{megabyte} di RAM, abbiamo deciso di 
avvalerci di LXDE, un ambiente desktop estremamente leggero, in modo da 
velocizzare la maggior parte delle nostre interazioni con il sistema. \\
Essendo tre persone a lavorare su una sola piattaforma, abbiamo optato 
per collocare la \emph{board} in un laboratorio dell'università dotandola
di un sistema di controllo remoto basato su \emph{virtual network computing} 
(VNC), un sistema che utilizza un protocollo \emph{remote framebuffer} (RBF) 
l'accesso remoto alle interfacce grafiche utente (meglio conosciute come 
\textbf{GUI}s, \textbf{G}raphical \textbf{U}ser \textbf{I}nterfaces). Inoltre, 
per facilitare l'accesso ai dati presenti sulle periferiche di memorizzazione, 
abbiamo dotato il Banana Pi di un server \emph{file transfer protocol} (FTP).
%----------------------------------------------------------------------------------------
%	SECTION 2
%----------------------------------------------------------------------------------------

\section{I tool per valutare le prestazioni}

%----------------------------------------------------------------------------------------
%	SECTION 3
%----------------------------------------------------------------------------------------

\section{I software di encoding presi in considerazione}

%----------------------------------------------------------------------------------------
%	SECTION 4
%----------------------------------------------------------------------------------------

\section{Test preliminari}

%----------------------------------------------------------------------------------------
%	SECTION 5
%----------------------------------------------------------------------------------------

\section{Individuazione dei moduli H.265}

%----------------------------------------------------------------------------------------
%	SECTION 6
%----------------------------------------------------------------------------------------

\section{Strategie di ottimizzazione}
% Chapter Template

\chapter{Risultati} % Main chapter title

\label{Chapter7} % Change X to a consecutive number; for referencing this chapter elsewhere, use \ref{ChapterX}

\lhead{Capitolo 7. \emph{Risultati}} % Change X to a consecutive number; this 
%is for the header on each page - perhaps a shortened title

Completato il progetto sono stati dedicati gli ultimi giorni ai vari 
\emph{benchmark} sulle singole tecniche di ottimizzazione applicate. 
I test sono stati fatti sullo stesso file, utilizzando ogni volta i seguenti 
due file di configurazione creati appositamente.\\

Il primo specifica il percorso completo alla sequenza d'ingresso, inclusi 
i dati su risoluzione, \emph{frame rate}, formato dei dati e numero di 
\emph{frame} da codificare.

\begin{lstlisting}
#=========================== File I/O ============================
InputFile          : /home/cpc/ssd/samples/yuv/highway_cif.yuv
InputBitDepth      : 8      # Input bitdepth
InputChromaFormat  : 420    # Ratio of luma to cr samples
FrameRate          : 30     # Frame Rate per second
FrameSkip          : 0      # Num of frames to be skipped in input
SourceWidth        : 352    # Input frame width
SourceHeight       : 288    # Input frame height
FramesToBeEncoded  : 72     # Number of frames to be coded
 
Level              : 3.1
\end{lstlisting}

Il secondo invece, omesso per brevità, contiene tutti i parametri 
dell'\emph{encoder} lasciati al valore di \emph{default}, fatta eccezione per 
il GOP. Infatti, a puro scopo di test, il GOP è stato ridefinito in questo  
modo: il primo frame è sempre \emph{intra-predicted}, ed è seguito dalla 
struttura  
``PBBB''
ripetuta 3 volte, dove P sta per \textit{predicted} e B per 
\textit{bipredicted}.\\
Verranno ora riassunti i risultati ottenuti. 
\\ \\
\textbf{Versione originale}\\
  Scaricato il codice sorgente dal \emph{repository} ufficiale di HM, è stato 
  subito compilato senza alcuna modifica al fine di valutarne le prestazioni 
  \emph{as-is}. Il tempo medio impiegato a codificare $72$ \emph{frame} di 
  \verb|highway_cif.yuv| secondo le modalità specificate sopra è risultato pari 
  a $1552.86$s.
\\ \\
\textbf{Opzioni di compilazione}\\
  Dopo aver compreso ed aggiunto le opzioni di compilazione
  si è ottenuto, in media, un miglioramento pari a $144.43$s, per un tempo 
  totale di $1408.43$s 
  (\textit{speed-up} $1.102$).
  La versione con le opzioni di compilazione è diventata poi la base per tutti 
  gli altri test, ogni benchmark è quindi da confrontarsi con quello descritto 
  in questo paragrafo.
\\ \\
\textbf{Assembly}\\
  I risultati delle funzioni ottimizzate a mano in \emph{assembly}, nonostante 
  queste abbiano avuto un discreto successo all'interno del codice di test, non 
  sono stati soddisfacenti in HM. %niente commenti personali qui, vanno nel cap 
  %8
  La durata media si è allungata di $85$s, arrivando a $1493.62$s. Questa 
  opzione è stata quindi scartata.
\\ \\
\textbf{Intrinsic Neon}\\
  Di tutte le strategie applicate, l'utilizzo delle SIMD di ARM ha prodotto i 
  migliori risultati sulle singole funzioni. Avendo inoltre avuto cura di 
  scegliere come obiettivo le più pesanti, il tempo complessivo ha subito un 
  discreto abbassamento.
  L'ottimizzazione delle funzioni dedicate al calcolo delle SAD ha accorciato i 
  tempi di $81.89$s, quella sulle funzioni dedicate al calcolo della WHT di 
  altri $99.98$s. Il tempo complessivo è quindi sceso a $1226.56$s   
  (\textit{speed-up}: $1.148$).
\\ \\
\textbf{Switch}\\
  La sostituzione degli \verb|if| nella funzione \verb|filter| ha portato a 
  vantaggi relativamente bassi. Più precisamente, la modifica ha migliorato 
  l'esecuzione di $11.14$s, abbassando il tempo complessivo a $1397.29$s.
\\ \\
\textbf{Attributi}\\
  L'aggiunta di attributi al codice in alcuni casi ha allungato il tempo di 
  esecuzione, in altri invece lo ha accorciato, mantenendo la media invariata 
  rispetto alla versione di riferimento, 
  indice del fatto che non ha apportato cambiamenti significativi. 
  Per questo motivo, la modifica è stata accantonata.
\\ \\
\textbf{Multithreading}\\
  Il passaggio da un singolo \emph{tread} a due, secondo le modalità 
  specificate nel capitolo dedicato, ha portato grossolanamente ad un raddoppio 
  delle prestazioni del programma. La durata dell'esecuzione si è accorciata di 
  $666.87$s, arrivando a $741.56$s (\textit{speed-up}: $1.899$).
\\ \\
\textbf{Prestazioni complessive}\\
  Di tutte le modifiche effettuate, che risiedono in \textit{branch} separati 
  nella \textit{repository} su GitHub, sono state selezionate ed inglobate 
  nella versione finale solo 
  quelle che hanno portato a miglioramenti nelle prestazioni.
  Come risultato il tempo complessivo si è abbassato dagli originali 
  $1552.86$s a $660.92$s, ottenendo quindi uno \textit{speed-up} pari a $2.351$.
  
%TODO IF

%HAD    da 198,803,125,750 a 116,394,523,093
%filter da 298,318,350,900 a 282,227,659,240
%SAD    da 122,866,807,420 a 23,543,068,820 
%Total  da 935,922,071,690 a 798,103,381,516
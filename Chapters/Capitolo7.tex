% Chapter Template

\chapter{Risultati} % Main chapter title

\label{Chapter7} % Change X to a consecutive number; for referencing this chapter elsewhere, use \ref{ChapterX}

\lhead{Capitolo 7. \emph{Risultati}} % Change X to a consecutive number; this 
%is for the header on each page - perhaps a shortened title

Completato il progetto è sono stati dedicati gli ultimi giorni ai vari 
\emph{benchmark} sulle singole tecniche di ottimizzazione applicate. 
I test sono stati fatti sullo stesso file, utilizzando ogni volta i seguenti 
due file di configurazione creati appositamente.\\

Il primo file specifica il percorso completo alla sequenza d'ingresso, inclusi 
i dati su risoluzione, \emph{frame rate}, formato dei dati e numero di 
\emph{frame} da codificare.

\begin{lstlisting}
  #======== File I/O ===============
  InputFile          : /home/cpc/ssd/samples/yuv/highway_cif.yuv
  InputBitDepth      : 8      # Input bitdepth
  InputChromaFormat  : 420    # Ratio of luminance to chrominance samples
  FrameRate          : 30     # Frame Rate per second
  FrameSkip          : 0      # Number of frames to be skipped in input
  SourceWidth        : 352    # Input frame width
  SourceHeight       : 288    # Input frame height
  FramesToBeEncoded  : 72     # Number of frames to be coded
  
  Level              : 3.1
\end{lstlisting}

Il secondo file invece specifica tutti i parametri dell'\emph{encoder}. A puro 
scopo di test è stato definito un GOP in questo modo: il primo frame è sempre 
un'Intra, seguito dalla seguente struttura ripetuta 3 volte (PBBB). 
L'intervallo tra due Intra è dunque 12. \\

Verranno ora discussi i vari risultati.

\paragraph{Versione originale \\}
  Scaricato il codice sorgente dal \emph{repository} ufficiale di HM, è stato 
  subito compilato senza alcuna modifica al fine di valutarne le prestazioni 
  \emph{as-is}. Il tempo medio impiegato a codificare $72$ \emph{frame} di 
  \verb|highway_cif.yuv| secondo le modalità specificate sopra è risultato pari 
  a $1552.86$ s.

\paragraph{Opzioni di compilazione \\}
  Dopo aver dedicato il tempo necessario al test delle opzioni di compilazione, 
  i risultati ottenuti sono stati i seguenti: in media si è ottenuto un 
  miglioramento pari a $144.43$ s, portando il tempo totale a $1408.43$ s 
  (\textit{speed-up}: $1.102$).
  
  La versione con le opzioni di compilazione è diventata poi la base per tutti 
  gli altri test, ogni tempo è quindi da confrontarsi con quello descritto in 
  questo paragrafo.
  
\paragraph{Assembly \\}
  I risultati delle funzioni ottimizzate a mano in \emph{assembly}, sebbene 
  siano state un'ottima esperienza dal punto di vista didattico, purtroppo non 
  sono stati soddisfacenti. 
  
  La durata media si è allungata di $85$ s, arrivando a $1493.62$ s. Questa 
  opzione è stata quindi scartata.
  
\paragraph{Intrinsic NEON \\}
  Di tutte le strategie applicate, l'utilizzo delle SIMD di ARM ha prodotto i 
  migliori risultati sulle singole funzioni. Avendo inoltre avuto cura di 
  scegliere come obiettivo le più pesanti, il tempo complessivo ha subito un 
  discreto abbassamento.
  
  L'ottimizzazione delle funzioni dedicate al calcolo delle SAD ha accorciato i 
  tempi di $81.89$ s, quella sulle funzioni dedicate al calcolo della WHT di 
  altri $99.98$ s. Il tempo complessivo è quindi sceso a $1226.56$ s   
  (\textit{speed-up}: $1.148$).
  
  // Instruction Fetch
  
\paragraph{Switch \\}
  La sostituzione degli \verb|if| ha portato a vantaggi relativamente bassi in 
  positivo, di seguito i risultati relativi ad essa.
  
  La modifica ha migliorato l'esecuzione di soli $11.14$ s, abbassando il tempo 
  complessivo a $1397.29$ s.
  
\paragraph{Attributi \\}
  L'aggiunta di attributi al codice in alcuni casi peggiorava drasticamente il 
  tempo di esecuzione, in altri invece non lo cambiava affatto. La modifica è 
  stata infine accantonata, non avendo ottenuto risultati migliori della 
  versione con le sole opzioni di compilazione, presa come riferimento.
  
\paragraph{Multithread \\}
  Il passaggio da un singolo \emph{tread} a due, secondo le modalità 
  specificate nel capitolo 6.4.6, ha portato grossolanamente ad un raddoppio 
  delle prestazioni del programma.
  
  L'esecuzione si è accorciata di ben $666.87$ s, arrivando a $741.56$ s 
  (\textit{speed-up}: $1.899$).

\paragraph{Prestazioni complessive \\}
  Di tutte le modifiche effettuate, che risiedono in \textit{branch} separati 
  nella \textit{repository} su GitHub, sono state selezionate ed inglobate solo 
  quelle che avevano portato a miglioramenti nelle prestazioni.
  
  Come risultato finale, il tempo complessivo è sceso dagli originali $1552.86$ 
  s a $660.92$ s, ottenendo quindi uno \textit{speed-up} pari a $2.351$.
    
%TODO grafico

%HAD    da 198,803,125,750 a 116,394,523,093
%filter da 298,318,350,900 a 282,227,659,240
%SAD    da 122,866,807,420 a 23,543,068,820 
%Total  da 935,922,071,690 a 798,103,381,516
% Chapter Template

\chapter{Calcolatori e dati} % Main chapter title

\label{Chapter2} % Change X to a consecutive number; for referencing this chapter elsewhere, use \ref{ChapterX}

\lhead{Capitolo 2. \emph{Calcolatori e dati}} % Change X to a consecutive number; this is for the header on each page - perhaps a shortened title

Negli ultimi due secoli il concetto di ``calcolatore'', a cui è subentrato nel
lessico comune il termine \emph{computer}, si è ampiamente esteso. Nei primi
anni del XIX secolo vennero poste le basi concettuali del computer programmabile, 
il primo modernamente definibile Turing-completo, da parte di Charles Babbage 
(non a caso considerato il padre del computer\footnote{Halacy, Daniel Stephen (1970). 
Charles Babbage, Father of the Computer. Crowell-Collier Press. ISBN 0-02-741370-5})
e negli anni '30 del XX secolo proprio Alan Turing definì i principi del odierno computer. 
\\
Il calcolatore, dunque, passa dall'essere uno strumento usato per
eseguire semplici calcoli matematici (in questa categoria potrebbe rientrare
anche un abaco) a macchina capace di eseguire calcoli matematici anche molto 
complessi (a cui ci si riferisce talvolta con il termine \emph{elaboratore}).
\\

%----------------------------------------------------------------------------------------
%	SECTION 1
%----------------------------------------------------------------------------------------

\section{Acquisizione ed elaborazione dei dati}

Affinché un computer possa eseguire dei calcoli è necessario che disponga di
\emph{dati} su cui effettuarli. L'acquisizione di questi ultimi può avvenire
in diversi modi, con o senza l'intervento di un essere umano.
Nel secondo caso i dati vengono spesso ottenuti grazie alla trasduzione 
di parametri fisici, acquisiti da sensori, in segnali elettrici, successivamente
tradotti da un convertitore analogico-digitale in modo da ottenere valori che 
possano essere compresi da un calcolatore binario.

%-----------------------------------
%	SUBSECTION 1
%-----------------------------------
\subsection{Potenza di calcolo}
La potenza di calcolo di un computer non è una misura che può essere effettuata
in maniera assoluta, ma può essere ottenuta, con approssimazione, come il 
risultato della sovrapposizione di diversi fattori:
\begin{itemize}
\item Potenza \emph{grezza} del processore (MIPS, MFLOPS)
\item Latenza degli accessi in memoria
\item Latenza delle interfacce I/O
\item Bandwidth
\end{itemize}
%-----------------------------------
%	SUBSECTION 2
%-----------------------------------

\subsection{Ottimizzazione di algoritmi}

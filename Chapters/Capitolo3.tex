% Chapter Template

\chapter{La necessità di comprimere i dati} % Main chapter title

\label{Chapter3}

\lhead{Capitolo 3. \emph{La necessità di comprimere i dati}}

%-------------------------------------------------------------------------------
%	SECTION 1
%-------------------------------------------------------------------------------

\section{Compressione dati}
La disponibilità di calcolatori sempre più potenti ad un costo sempre più
contenuto ha portato alla conseguente possibilità di effettuare elaborazioni
via via più complesse, incrementando di pari passo la mole di dati su cui
queste elaborazioni vengono svolte. \\
(Per fare un esempio) nel 1988 venne proposto, all'interno di H.261, lo standard
\emph{Common Intermediate Format} (CIF) per uniformare la risoluzione verticale
ed orizzontale delle sequenze video solitamente utilizzate nelle teleconferenze.
Lo standard prevede una risoluzione pari a 352${\times}$288 pixel ed un
\emph{frame rate} pari a 30000/1001 (circa 29.97) \emph{frames} al secondo;
prendendo in considerazione una codifica senza compressione in uno spazio
colore RGB con 8 bit per canale ed applicando la formula
\begin{align*}
  \text{MB}\!/\!\text{s} = 
  W \times H \times bit\_depth \times frames\!/\!\text{s} \:/\: 1000
\end{align*}
si ottiene che un secondo di sequenza video pesa circa 9 \emph{megabyte}. \\
Considerando la capacità tipica di un disco rigido dell'epoca (compresa
tra i 20 e i 60MB$^{[citazione\: necessaria]}$) o la larghezza di banda 
necessaria a trasmettere una tale sequenza, risulta evidente come la 
compressione dei dati sia stata una grande necessità; se effettuiamo lo stesso 
calcolo utilizzando una risoluzione standard moderna (e.g., 1920$\times$1080), 
otteniamo un peso di circa 187MB per ogni secondo contenuto nella sequenza 
video. \\
Sebbene a tutt'oggi le capacità di memorizzazione siano incrementate di fattori
che variano tra le 12'500 e le 200'000 volte\footnote{Sono state confrontate
rispettivamente capacità di 500GB e 8TB con la capacità media di 40MB 
dell'epoca.}, la velocità di trasmissione dei dati non è aumentata di pari 
passo; la compressione dei dati rimane comunque anche un'ottima risorsa per
poter immagazzinare grandi quantità di dati con la possibilità di preservare
la qualità originale di questi ultimi.
%-----------------------------------
%	SUBSECTION 1
%-----------------------------------
\subsection{Compressione \emph{lossless}}

%-----------------------------------
%	SUBSECTION 2
%-----------------------------------

\subsection{Compressione \emph{lossy}}

%-------------------------------------------------------------------------------
%	SECTION 2
%-------------------------------------------------------------------------------

\section{Compressione di immagini}

%-----------------------------------
%       SUBSECTION 1
%-----------------------------------

\subsection{Il sistema visivo umano}

%-----------------------------------
%       SUBSECTION 2
%-----------------------------------

\subsection{Utilizzo delle trasformate}

%-----------------------------------
%       SUBSECTION 3
%-----------------------------------

\subsection{Principali algoritmi}

%-------------------------------------------------------------------------------
%       SECTION 3
%-------------------------------------------------------------------------------

\section{Compressione video}

%-----------------------------------
%       SUBSECTION 1
%-----------------------------------

\subsection{Ridondanza spaziale}

%-----------------------------------
%       SUBSECTION 2
%-----------------------------------

\subsection{Ridondanza temporale}

%-----------------------------------
%       SUBSECTION 3
%-----------------------------------

\subsection{Principali algoritmi}


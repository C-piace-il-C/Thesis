% Chapter Template

\chapter{La necessità di comprimere i dati} % Main chapter title

\label{Chapter3}

\lhead{Capitolo 3. \emph{La necessità di comprimere i dati}}

%-------------------------------------------------------------------------------
%	SECTION 1
%-------------------------------------------------------------------------------

\section{Compressione dati}

%-----------------------------------
%	SUBSECTION 1
%-----------------------------------
\subsection{Compressione \emph{lossless}}

Gli algoritmi di compressione senza perdita (\emph{lossless}) solitamente 
sfruttano le ridondanze per rappresentare i dati nel modo più sintetico 
possibile senza andare ad intaccare il messaggio originale: senza cioè alcuna 
perdita d'informazione.\\

Il limite di questa classe di algoritmi è definito dal primo teorema di 
Shannon, che definisce il significato operativo di entropia e vincola la 
massima compressione possibile.
Eccetto alcuni casi particolari è estremamente dispendioso in termini 
computazionali avvicinarsi esattamente al limite teorico della compressione.\\

Queste tecniche sono necessarie laddove non è ammessa la corruzione del dato 
originale: compressione di documenti e programmi ma anche di audio e video ad 
alta qualità, per applicazioni professionali o di estrazione d'informazione da 
parte di un calcolatore.

%-----------------------------------
%	SUBSECTION 2
%-----------------------------------

\subsection{Compressione \emph{lossy}}

Se l'applicazione non richiede una ricostruzione esatta del messaggio originale 
è possibile utilizzare più efficienti algoritmi di compressione con perdita 
(\emph{lossy}).
Questa classe di algoritmi ha solitamente come fruitore del risultato l'uomo. 
In questo caso vengono sfruttate le conoscenze che si hanno dell'apparato 
audio-visivo al fine di rendere impercettibile la perdita d'informazione, in 
aggiunta a tutte le tecniche di compressione \emph{lossless}.
La resa in termini di dimensioni del compresso è molto superiore rispetto alla 
variante senza perdita. Nel caso di audio ed immagini la dimensione del dato di 
uscita rispetto all'originale passa in media dal $50\%$ della variante 
\emph{lossless} al $10\%$ se viene utilizzata una compressione \emph{lossy}.

%-------------------------------------------------------------------------------
%	SECTION 2
%-------------------------------------------------------------------------------

\section{Compressione di immagini}

% \subsection{Il sistema visivo umano}

%-----------------------------------
%       SUBSECTION 1
%-----------------------------------

\subsection{Utilizzo delle trasformate}

%-----------------------------------
%       SUBSECTION 2
%-----------------------------------

\subsection{Principali algoritmi}

%-------------------------------------------------------------------------------
%       SECTION 3
%-------------------------------------------------------------------------------

\section{Compressione video}

%-----------------------------------
%       SUBSECTION 1
%-----------------------------------

\subsection{Ridondanza spaziale}

%-----------------------------------
%       SUBSECTION 2
%-----------------------------------

\subsection{Ridondanza temporale}

%-----------------------------------
%       SUBSECTION 3
%-----------------------------------

\subsection{Principali algoritmi}


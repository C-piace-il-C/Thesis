% Chapter Template

\chapter{La necessità di comprimere i dati} % Main chapter title

\label{Chapter3}

\lhead{Capitolo 3. \emph{La necessità di comprimere i dati}}

%-------------------------------------------------------------------------------
%	SECTION 1
%-------------------------------------------------------------------------------

\section{Compressione dati}

%-----------------------------------
%	SUBSECTION 1
%-----------------------------------
\subsection{Compressione \emph{lossless}}

Gli algoritmi di compressione senza perdita (\emph{lossless}) solitamente 
sfruttano le ridondanze per rappresentare i dati nel modo più sintetico 
possibile senza andare ad intaccare il messaggio originale: senza cioè alcuna 
perdita d'informazione.\\

Il limite di questa classe di algoritmi è definito dal primo teorema di 
Shannon, che definisce il significato operativo di entropia e vincola la 
massima compressione possibile.
Eccetto alcuni casi particolari è estremamente dispendioso in termini 
computazionali avvicinarsi esattamente al limite teorico della compressione.\\

Queste tecniche sono necessarie laddove non è ammessa la corruzione del dato 
originale: compressione di documenti e programmi ma anche di audio e video ad 
alta qualità, per applicazioni professionali o di estrazione d'informazione da 
parte di un calcolatore.

%-----------------------------------
%	SUBSECTION 2
%-----------------------------------

\subsection{Compressione \emph{lossy}}

%-------------------------------------------------------------------------------
%	SECTION 2
%-------------------------------------------------------------------------------

\section{Compressione di immagini}

%-----------------------------------
%       SUBSECTION 1
%-----------------------------------

\subsection{Il sistema visivo umano}

%-----------------------------------
%       SUBSECTION 2
%-----------------------------------

\subsection{Utilizzo delle trasformate}

%-----------------------------------
%       SUBSECTION 3
%-----------------------------------

\subsection{Principali algoritmi}

%-------------------------------------------------------------------------------
%       SECTION 3
%-------------------------------------------------------------------------------

\section{Compressione video}

%-----------------------------------
%       SUBSECTION 1
%-----------------------------------

\subsection{Ridondanza spaziale}

%-----------------------------------
%       SUBSECTION 2
%-----------------------------------

\subsection{Ridondanza temporale}

%-----------------------------------
%       SUBSECTION 3
%-----------------------------------

\subsection{Principali algoritmi}


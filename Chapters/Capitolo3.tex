% Chapter Template

\chapter{La necessità di comprimere i dati} % Main chapter title

\label{Chapter3}

\lhead{Capitolo 3. \emph{La necessità di comprimere i dati}}

%-------------------------------------------------------------------------------
%	SECTION 1
%-------------------------------------------------------------------------------

\section{Compressione dati}

%-----------------------------------
%	SUBSECTION 1
%-----------------------------------
\subsection{Compressione \emph{lossless}}

Gli algoritmi di compressione senza perdita (\emph{lossless}) solitamente 
sfruttano le ridondanze per rappresentare i dati nel modo più sintetico 
possibile senza andare ad intaccare il messaggio originale: senza cioè alcuna 
perdita d'informazione.\\

Il limite di questa classe di algoritmi è definito dal primo teorema di 
Shannon, che definisce il significato operativo di entropia e vincola la 
massima compressione possibile.
Eccetto alcuni casi particolari è estremamente dispendioso in termini 
computazionali avvicinarsi esattamente al limite teorico della compressione.\\

Queste tecniche sono necessarie laddove non è ammessa la corruzione del dato 
originale: compressione di documenti e programmi ma anche di audio e video ad 
alta qualità, per applicazioni professionali o di estrazione d'informazione da 
parte di un calcolatore.

%-----------------------------------
%	SUBSECTION 2
%-----------------------------------

\subsection{Compressione \emph{lossy}}

Se l'applicazione non richiede una ricostruzione esatta del messaggio originale 
è possibile utilizzare più efficienti algoritmi di compressione con perdita 
(\emph{lossy}).
Questa classe di algoritmi ha solitamente come fruitore del risultato l'uomo. 
In questo caso vengono sfruttate le conoscenze che si hanno dell'apparato 
audio-visivo al fine di rendere impercettibile la perdita d'informazione, in 
aggiunta a tutte le tecniche di compressione \emph{lossless}.
La resa in termini di dimensioni del compresso è molto superiore rispetto alla 
variante senza perdita. Nel caso di audio ed immagini la dimensione del dato di 
uscita rispetto all'originale passa in media dal $50\%$ della variante 
\emph{lossless} al $10\%$ se viene utilizzata una compressione \emph{lossy}.

%-------------------------------------------------------------------------------
%	SECTION 2
%-------------------------------------------------------------------------------

\section{Compressione di immagini}

% \subsection{Il sistema visivo umano}

%-----------------------------------
%       SUBSECTION 1
%-----------------------------------

\subsection{Utilizzo delle trasformate}

I metodi più diffusi per quanto riguarda la compressione di immagini prevedono 
l'utilizzo di trasformate con lo scopo di rendere l'immagine più adatta ad 
essere codificata.
Essa viene prima divisa in blocchi di dimensioni ridotte, $N \cdot N$, 
solitamente quadrate e non superiori a $64 \cdot 64$.\\

Su ogni blocco viene effettuata separatamente una trasformata, il cui obiettivo 
è quello di de-correlare il più possibile il segnale originale. La bontà di una 
trasformata viene solitamente definita in base alle sue capacità di 
de-correlazione e d'implementazione veloce.\\

Tra le trasformate si ricordano:

\begin{itemize}
  
  \item \textbf{\emph{Discrete Fourier transform} (DFT)}\\
    Largamente utilizzata in analisi e filtraggio, ha la proprietà di avere un 
    nucleo separabile (rendendo possibile quindi il calcolo della trasformata 
    isolatamente su righe e colonne). Ha una sua implementazione veloce, 
    \emph{Fast Fourier transform} (FFT), che rende il suo utilizzo molto 
    appetibile.
    
  \item \textbf{\emph{Karhunen–Loève transform} (KLT)}\\
    Fornisce il miglior compattamento d'energia rispetto alle altre 
    trasformazioni possibili. Purtroppo la mancanza di un algoritmo veloce e la 
    necessità di uno studio della covarianza dell'immagine da comprimere per 
    generare le funzioni base la rendono scarsamente utilizzata.
    
  \item \textbf{\emph{Discrete Cosine transform} (DCT)}\\
    Molto simile alla DFT, utilizza come funzioni base solo coseni.
    Permette di ottenere una de-correlazione molto simile a quella ottenibile 
    trasformando con KLT. La presenza di un algoritmo veloce la rende tutt'ora 
    la trasformata più utilizzata per la compressione d'immagini.
   
  \item \textbf{\emph{Walsh-Hadamard transform (WHT)}}\\
    La peggiore in termini di compattamento d'energia, ha la proprietà di poter 
    essere eseguita con sole somme e sottrazioni ed ha una sua versione 
    \emph{fast}. Il bassissimo costo computazionale l'ha resa largamente 
    utilizzata.
    
\end{itemize}

%-----------------------------------
%       SUBSECTION 2
%-----------------------------------

\subsection{Principali algoritmi}

%-------------------------------------------------------------------------------
%       SECTION 3
%-------------------------------------------------------------------------------

\section{Compressione video}

%-----------------------------------
%       SUBSECTION 1
%-----------------------------------

\subsection{Ridondanza spaziale}

%-----------------------------------
%       SUBSECTION 2
%-----------------------------------

\subsection{Ridondanza temporale}

%-----------------------------------
%       SUBSECTION 3
%-----------------------------------

\subsection{Principali algoritmi}


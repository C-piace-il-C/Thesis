% Chapter Template

\chapter{Conclusioni e sviluppi futuri} % Main chapter title
In questa Tesi sono state affrontate più tematiche di grande attualità  
nell'ottica dell'ottimizzazione del software, quali la compressione 
dell'informazione e, soprattutto, l'encoding video, facendo riferimento allo 
standard di 
video 
compressione più recente: il 
\emph{High Efficiency Video Coding} (HEVC) o H.265, sviluppato dalla 
collaborazione tra il \emph{Moving Picture Experts Group} (MPEG) ed il 
\emph{Video Coding Experts Group} (VCEG). Lo standard HEVC promette di 
raddoppiare il 
rapporto di compressione delle sequenze rispetto al predecessore H.264, 
migliorandone al contempo la qualità.\\ \\
L'encoder scelto come riferimento, nello specifico HEVC Test 
Model (HM) sviluppato da Fraunhofer, è stato ottimizzato per il 
funzionamento sulla piattaforma embedded ARM Banana Pi, aprendo la strada verso 
modifiche \emph{system dependent}.\\ \\
I candidati hanno sperimentanto diverse ottimizzazioni cercando di coinvolgere 
maggiormente le risorse disponibili nella piattaforma scelta, avvicinandosi 
anche alla programmazione a basso livello. Tra le ottimizzazioni proposte 
spiccano 
parallelizzazioni 
quali calcoli attraverso le intrinsic NEON ed il multithreading. \\
Il lavoro è stato svolto collaborando in remoto su un \emph{repository} 
git pubblicamente accessibile\footnote{https://github.com/C-piace-il-C/} 
ospitato su GitHub, i candidati 
hanno pertanto avuto modo di acquisire esperienza sugli strumenti relativi al 
lavoro collaborativo oltre che sull'ambiente Linux, sull'architettura ARM e 
sulla programmazione in generale.\\ \\
L'opinione predominante degli autori alla luce del lavoro svolto è che le 
ottimizzazioni implementate, che si rifanno principalmente al tentativo di 
sfruttare al meglio la potenza bruta del Banana Pi, evidenziano che la carenza 
nel software di HM si trova altrove: il confronto con il progetto di x265 
rafforza l'ipotesi che il problema sia nell'organizzazione strutturale del 
codice.\\ \\
Come conseguenza, possibili sviluppi futuri consistono in un'analisi più 
accurata ed approfondita della logica nel funzionamento complessivo di HM alla 
ricerca di ridondanze e organizzazioni inefficienti o  
disfunzionali (?) nell'architettura del codice, anche se è impossibile 
prevedere la mole delle modifiche che tale analisi rivelerebbe.
\label{Chapter8} % Change X to a consecutive number; for referencing this chapter elsewhere, use \ref{ChapterX}

\lhead{Capitolo 8. \emph{Conclusioni e sviluppi futuri}} % Change X to a 
%consecutive number; this is for the header on each page - perhaps a shortened 
%title 
% Chapter Template

\chapter{Introduzione} % Main chapter title

\label{Chapter1}

\lhead{Chapter 1. \emph{Introduzione}}

%-------------------------------------------------------------------------------
%	SECTION 1
%-------------------------------------------------------------------------------

La compressione di sequenze video è un annoso problema che riguarda diversi
ambiti che spaziano dalla semplice memorizzazione alla trasmissione analogica
o digitale delle suddette; negli ultimi anni il progressivo aumento
della risoluzione media dei dispositivi di riproduzione (e dunque, seppur in
misura minore, dei contenuti) ha portato alla necessità di sviluppare nuovi
algoritmi che permettessero un migliore rapporto di compressione e sfruttassero
al meglio l'architettura parallelizzata dei calcolatori moderni. \\
Questo ha portato alla nascita del soggetto in esame di questa tesi, ovvero
H.265 (meglio noto come \emph{High Efficiency Video Coding}, o HEVC), un 
algoritmo che si pone come obiettivi un miglioramento dell'efficienza e
della compressione rispetto al suo predecessore, H.264, e il supporto a
risoluzioni fino a $8192{\times}4320$\citep{OverHEVC}. 
\\ \\
L'ottimizzazione di tali algoritmi risulta utile di per sé a ridurre il tempo
impiegato alla codifica e alla decodifica delle sequenze, portando spesso ad
un risparmio di tipo economico, ma copre un ruolo fondamentale in caso di
codifiche in tempo reale, nelle quali sono previsti specifici limiti inferiori
delle prestazioni richieste (in questi casi tipicamente un filmato deve essere 
codificato ad una velocità non minore di 24 fotogrammi al secondo). 
\\ \\
La scelta di ottimizzare un algoritmo ``giovane'' come H.265 è scaturita sia
dalla volontà di lavorare su uno strumento che è destinato ad essere un punto
di riferimento delle codifiche video del presente e del futuro imminente, sia
dalla possibilità di lavorare su del \emph{software} che non possiede ancora
ottimizzazioni specifiche per diverse architetture (come possiedono invece
molte implementazioni di H.264, ratificato nel Marzo 2003), permettendoci
dunque di effettuare diversi miglioramenti di validità generale potendo comunque
apprezzare i risultati, seppur minimi.
\\ \\
La struttura rimanente di questa Tesi è organizzata secondo il seguente schema:
\begin{itemize}
\item Il Capitolo 2 presenta una breve storia dei calcolatori, fino ad arrivare
      a descrivere il contesto odierno nel quale è stato svolto questo progetto;
\item Il Capitolo 3 offre una visione d'insieme più esaustiva di questa 
      introduzione sulla necessità di comprimere dati, descrivendo le tecniche
      più comuni con cui ciò viene realizzato;
\item Il Capitolo 4 cerca di descrivere accuratamente la struttura e il
      funzionamento dell'algoritmo H.265;
\item Il Capitolo 5 descrive le caratteristiche della piattaforma su cui il
      progetto è stato svolto;
\item Il Capitolo 6 riporta dettagliatamente il percorso seguito nello 
      svolgimento del progetto e i diversi approcci che sono stati tentati per
      ottenere risultati soddisfacenti;
\item Il Capitolo 7 contiene i risultati ottenuti applicando le strategie
      descritte nel Capitolo 6;
\item Il Capitolo 8, infine, presenta le conclusioni a cui si è arrivati ed i
      possibili sviluppi futuri dell'argomento preso in esame.
\end{itemize}